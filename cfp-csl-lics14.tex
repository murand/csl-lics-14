\documentclass[oneside]{article}
\usepackage[letterpaper,margin={12mm,9mm}]{geometry}
\usepackage{times,courier}
\usepackage{xcolor}
\colorlet{hrefcolor}{blue!30!white}
\usepackage{hyperref}
\hypersetup{
  colorlinks=false,
  urlbordercolor=hrefcolor,
  pdfborderstyle={/S/U/W 1},
}
\usepackage{microtype}

\parindent 0pt
\parskip 0pt

\begin{document}

\pagestyle{empty}
\raggedbottom

\newbox\nbox
\newcommand\titlestyle{\large\bfseries}

\begin{center}
  \textbf{CALL FOR PAPERS} \\[1ex]
  {\large JOINT MEETING OF} \\[1ex]
  \setbox\nbox=\hbox{\titlestyle COMPUTER SCIENCE LOGIC (CSL)}
  \begin{minipage}[b]{\wd\nbox} \centering
    the Twenty-Third EACSL Annual Conference on \\[.2ex]
    {\box\nbox}
  \end{minipage}
  \ {\titlestyle\ \&\ }
  \setbox\nbox=\hbox{\titlestyle LOGIC IN COMPUTER SCIENCE (LICS)}
  \begin{minipage}[b]{\wd\nbox} \centering
    the Twenty-Ninth Annual ACM/IEEE Symposium on \\[.2ex]
    {\box\nbox}
  \end{minipage}
  \\[1ex]
  July 14--18, 2014, Vienna, Austria \\[1ex]
  {\bfseries \url{http://vsl2014.at/csl-lics/}
             \quad
             \url{http://lii.rwth-aachen.de/lics/csl-lics14/}
  }
\end{center}

\vspace{1mm}

\newdimen\leftcol
\leftcol=4.7cm
\newdimen\rightcol
\rightcol=\linewidth
\advance\rightcol by -\leftcol
\advance\rightcol by -1.5em
\let\oldframebox=\framebox
\let\framebox=\relax

%%% TODO uncomment the following two lines to display sizing boxes
% \fboxsep 0pt
% \let\framebox=\oldframebox

\framebox{
\begin{minipage}[t]{\leftcol}
  % \linespread{1.07}
  \parskip .4ex
  \small
  \textbf{Program Chairs:} \\[.4ex]
  Thomas Henzinger, \emph{IST Austria}\\
%\\  \href{mailto:tah@ist.ac.at?subject=CSL/LICS+2014}
%         {\ttfamily tah@ist.ac.at} \\[.4ex]
  Dale Miller, \emph{INRIA \& LIX}
% \\  \href{mailto:dale.miller@inria.fr?subject=CSL/LICS+2014}
%          {\ttfamily dale.miller@inria.fr}

  \vspace{.5em}

  \textbf{Program Committee:} \\[.4ex]
  A.\ Asperti, \emph{U. Bologna} \\
  G.\ Barthe, \emph{IMDEA} \\
  A.\ Bauer, \emph{IMFM} \\
  L.\ Birkedal, \emph{Aarhus U.} \\
  K.\ Chatterjee, \emph{IST Austria} \\
  A.\ Compagnoni, \emph{Stevens Inst. of Tech.} \\
  V.\ De Paiva, \emph{Nuance Communications} \\
  L.\ Doyen, \emph{ENS Cachan} \\
  J.\ Duparc, \emph{U. Lausanne} \\
  M.\ Fernandez, \emph{King's College London} \\
  H.\ Geuvers, \emph{Radboud U. Nijmegen} \\
  D.\ Ghica, \emph{U. Birmingham} \\
  E.\ Gr\"adel, \emph{RWTH Aachen U.} \\
  H.\ Hermanns, \emph{Saarland U.} \\
  N.\ Immerman, \emph{U. Mass. Amherst} \\
  N.\ Kobayashi, \emph{U. Tokyo} \\
  L.\ Kovacs, \emph{Chalmers U.} \\
  V.\ Kuncak, \emph{EPFL} \\
  S.\ La Torre, \emph{U. Salerno} \\
  R.\ Majumdar, \emph{MPI-SWS} \\
  D.\ Mazza, \emph{CNRS \& U. Paris-Nord} \\
  J.\ Ouaknine, \emph{U. Oxford} \\
  L.\ Pacholski, \emph{U. Wroclaw} \\
  N.\ Piterman, \emph{U. Leicester} \\
  A.\ Pitts, \emph{U. Cambridge} \\
  A.\ Podelski, \emph{U. Freiburg} \\
  R.\ Ramanujam, \emph{IMS Chennai} \\
  J.\ Riely, \emph{DePaul U.} \\
  S.\ Ronchi Della Rocca, \emph{U. Torino} \\
  A.\ Sabry, \emph{Indiana U.} \\
  T.\ Schrijvers, \emph{Ghent U.} \\
  P.\ S.\ Thiagarajan, \emph{Nat. U. Singapore} \\
  A.\ Tiu, \emph{Australian Nat. U.} \\
  V.\ Vianu, \emph{U. California San Diego} \\
  A.\ Voronkov, \emph{U. Manchester} \\
  I.\ Walukiewicz, \emph{CNRS \& U. Bordeaux}

  \vspace{.5em}

  \textbf{Workshop Chairs:} \\[.4ex]
  Patricia Bouyer-Decitre, \emph{CNRS} \\
  Georg Moser, \emph{U.\ Innsbruck}
%\\   \href{mailto:georg.moser@uibk.ac.at?subject=CSL/LICS+2014}
%    {\ttfamily georg.moser@uibk.ac.at}

  \vspace{.5em}

  \textbf{Local Organization Committee:}\\[.4ex]
  K.\ Chatterjee and J. Otop, \emph{IST Austria}

  \vspace{.5em}

  \textbf{Publicity Chairs:}\\[.4ex]
  K.\ Chaudhuri, % \emph{INRIA \& LIX}\\
  A.\ Murawski% , \emph{U. Warwick}

  \vspace{.5em}

  \textbf{FLoC Organization Committee:} \\[.4ex]
  M. Baaz, S. Szeider, M. Vardi,\\ H. Veith

%\vspace{.5em}

%  \textbf{Important Dates:} \\[.4ex]
%    \begin{tabular}[c]{rl@{\ }r@{,\ }l}
%      Title \& abstracts due: & Jan & 13 & 2014 \\
%      Full paper due:         & Jan & 20 & 2014 \\
%      Author notification:    & Mar & 31 & 2014 \\
%      Final versions due:     & May & 15 & 2014
%    \end{tabular}

  %%%% EDNOTE KC omitted for space: organizing committee, steering committe
\end{minipage}}
\hfill
%%% TODO KC remove framebox
\framebox{
\begin{minipage}[t]{\rightcol}
  \parindent 1.4em

  \noindent%
  CSL is the annual meeting of the \href{http://eacsl.org}{European Association
  for Computer Science Logic (EACSL)} intended for computer scientists whose
  research activities involve logic, as well as for logicians working on issues
  significant for computer science.
  %
  LICS is an annual international forum on theoretical and
  practical topics in computer science that relate to logic.
  %
  The organizers of these two series of meetings have chosen to join the
  2014 editions of these meetings into a single event within the Federated Logic
  Conference (FLoC) that will be part of the \href{http://vsl2014.at}{Vienna
  Summer of Logic 2014}.
  %
  Thus, in 2014, these meetings will have one program committee, one program,
  and one proceedings.  No decision has been made to hold CSL and LICS jointly
  beyond 2014.

  We invite submissions on topics that fit the themes of both conferences.
  %
  These topics include:
  %
  % written this way so you can run `sort-lines' on the region
automata theory;
automated deduction;
categorical models and logics;
constraints programming;
constructive mathematics;
database theory;
decision procedures;
domain theory;
finite model theory;
formal languages;
formal methods in software engineering;
foundations of computability;
functional and reactive synthesis;
game semantics;
graph games;
higher-order logic;
lambda and combinatory calculi;
linear logic;
logic programming;
logics for AI;
logics of programs;
logical aspects of computational complexity;
modal and temporal logics;
model checking;
program analysis;
proof theory;
semantics of programming languages;
specification and verification of hardware, software, and complex systems;
term rewriting; and
type theory.
Also welcome are papers describing models and logics for biological systems;
concurrent, distributed, and mobile computation;
quantum computation;
security; and
real-time, probabilistic, and hybrid systems.

  \medskip

  \noindent%
  \textbf{Submission:} %
  Authors are required to submit a paper title and a short abstract of about 100
  words in advance of submitting the full paper.
  %
  Every full paper must be submitted in the {IEEE} Proceedings 2-column
  10-point format and may not be longer than 10 pages, including references.
  ({\LaTeX} style files are available on the conference web-site.)
  %
  The full paper must be in English and provide sufficient detail to
  allow the program committee to assess its merits.
%
  Full proofs may appear in a technical appendix which will be read at the
  reviewers' discretion.
%
  Authors are strongly encouraged to include a well written introduction which
  is directed at all members of the program committee.
%
  Paper selection will be merit-based, with no \emph{a priori} limit on the number
  of accepted papers.
  %
%  It should begin with a succinct statement of the issues, a summary of the main
%  results, and a brief explanation of their significance and the relevance to
%  the conference and to computer science, all phrased for the non-specialist.
  %
%  Technical development directed to the specialist should follow.
  %
%  References and comparisons with related work must be included.
  %
%  If necessary, detailed proofs of technical results may be included in a
%  clearly labeled appendix, to be consulted at the discretion of program
%  committee members.
  %
%  full papers not conforming to the above requirements will be rejected
%  without further consideration.
%
   Papers authored or co-authored by members of the program committee are not
   allowed.

   \medskip

   \noindent%
   \textbf{Important Dates:}
% Authors are required to submit a paper title and a
% short abstract of about 100 words in advance of submitting the full paper of
% the paper.
The exact deadline time on these dates is given by
   \href{http://www.ieee802.org/16/aoe.html}{AoE} (``anywhere on earth'').
  %
  \begin{center}
    \hspace{-8em} \small
    \begin{tabular}[c]{rr@{\ }r@{,\ }l}
      \textbf{Title \& Short Abstracts Due:}
      & January & 13 & 2014 \\
      \textbf{Full Papers Due:}
      & January & 20 & 2014 \\
      \textbf{Author Notification:}
      & March   & 31 & 2014 \\
      \textbf{Final Versions Due for Proceedings:}
      & May     & 15 & 2014
    \end{tabular}
  \end{center}
  % \begin{center}
  % \begin{tabular}{rr}\small
  % {\bf Titles \& Short Abstracts Due:}      & January  13, 2014\\
  % {\bf Full Papers  Due:}            & January 20, 2014\\
  % {\bf Author Notification:}                & March 31, 2014\\
  % {\bf Final Versions Due for Proceedings:} & May 15, 2014
  % \end{tabular}
  % \end{center}
  %
  Deadlines are firm; late submissions will not be possible.
  %
  All submissions are made electronically via the url
  \textbf{\url{http://easychair.org/conferences/conf=csllics2014}}.

  The results reported in submissions must be unpublished and not submitted for
  publication elsewhere, including the proceedings of other symposia or workshops.
  %
  The program chairs must be informed in advance of submission of any closely
  related work submitted or about to be submitted to a conference or journal.
  %
  Authors of accepted papers are expected to sign copyright release forms.
  %
  One author of each accepted paper is expected to present that paper at the conference.

  \medskip

%  \noindent%
%  \textbf{Short Presentations:} %
%  A session of short presentations, intended for descriptions of student
%  research works in progress, and other brief communications, is planned.
%  %
%  These abstracts will not be published.
%  %
%  Dates and guidelines are posted on the conference web-site.
%
%  \smallskip

  \noindent%
  \textbf{Awards:} %
  The \emph{Kleene Award for Best Student Paper} will be given for the best
  student paper(s), as judged by the program committee.
  %
  The \emph{EACSL Outstanding Dissertation Award}, named for Wilhelm F.\
  Ackermann,
  and the \emph{LICS Test-of-Time Award 2014}
  will be presented during the joint meeting.

  \medskip

%  \noindent%
%  \textbf{Special Issue:} %
%  Full versions of up to three accepted papers, to be selected by the program
%  committee, will be invited for submission to the \emph{Journal of the ACM}.
%  %
%  Additional selected papers will be invited to a special issue of \emph{Logical
%    Methods in Computer Science}.
%
%  \smallskip

  \noindent%
  \textbf{Sponsorship:} %
  The joint meeting is sponsored by the European Association for Computer
  Science Logic, the IEEE Technical Committee on Mathematical Foundations of
  Computation, and by the ACM SIGACT in cooperation with the Association for
  Symbolic Logic and the European Association for Theoretical Computer Science.
  % %
  % The Vienna Summer of Logic 2014 is organized by the Kurt G\"odel Society.

 %  \emph{Additional space for names of various committee members if the PC is
%    longer than 44 people, logos, etc.}
\end{minipage}
}

%\bigskip
\medskip

 \noindent%
\bgroup \small
  \textbf{EACSL Executive Committee:}
L. Beklemishev, M. Bezem, B. Courcelle, A. Dawar, Z. Esik, B. Loewe,
J. Makowsky (president), D. Niwinski, L. Ong, T. Schwentick, H. Veith,
G. Winskel
%
%  \noindent%
\quad  \textbf{LICS Organizing Committee:}
M. Abadi, L. Aceto, R. Alur, F. Baader, P. Beame, P. Bouyer-Decitre,
K. Chatterjee, A. Compagnoni, A. Dawar, N. Dershowitz, M. Fernandez,
M. Grohe, O. Grumberg, T. Henzinger, P. Kolaitis, O. Kupferman,
B. Larose, V. Lipovac, D. Miller, M. Mislove, G. Moser, A. Murawski, L. Ong
(chair), A. Scedrov, D. Shmoys, M. Valeriote
\egroup

\end{document}

% Local Variables:
% mode: latex
% fill-column: 80
% sort-fold-case: t
% ispell-local-dictionary: "en_US"
% End:

%  LocalWords:  CSL EACSL LICS ACM IEEE Henzinger EPFL INRIA Givenname Vieille
%  LocalWords:  Terre Georg Moser FLoC Helmut Veith VSL Baaz automata logics rl
%  LocalWords:  computability combinatory bioinformatics symposia Kleene SIGACT
%  LocalWords:  Ackermann odel
